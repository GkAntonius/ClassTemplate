\documentclass[12pt,solution=false]{uqtrassignment}
\usepackage[utf8]{inputenc}
\usepackage{amsmath}
\usepackage{braket}
\usepackage{color}
\usepackage{graphicx}
\graphicspath{{img/}}

\coursecode{PMO1008}
\semester{Automne 2022}
\authormail{gabriel.antonius@uqtr.ca}
\department{Département de chimie, biochimie et physique}

\title{Exercice 1}
\author{Gabriel Antonius}

\setdocumentlabel{Exercice}

% ============================================================================ %

\begin{document}

\maketitle




\begin{problem}{Particule en 1D}

Considérez deux fonctions d'onde réelles
  pour une particule en une dimension
\begin{align}
  \psi_1(x) &= C_1 e^{-x^2}
\\
  \psi_2(x) &= C_2 x e^{-x^2}
\end{align}



\subproblem
Sachant que ces fonctions d'onde doivent être normalisées,
  que valent les constantes $C_1$ et $C_2$?

\subproblem
Considérez maintenant l'opérateur
\begin{align}
  W = a X + b X^2
\end{align}

Quelle est la représentation matricielle de $W$ dans la base des
  $\psi_1$ et $\psi_2$ ?
Autrement dit,
  construisez la matrice
\begin{align}
  W
  =
  \begin{bmatrix}
    W_{11}
    &
    W_{12}
    \\
    W_{21}
    &
    W_{22}
    \\
  \end{bmatrix}
  =
  \begin{bmatrix}
    \braket{\psi_1 \vert W \vert \psi_1}
    &
    \braket{\psi_1 \vert W \vert \psi_2}
    \\
    \braket{\psi_2 \vert W \vert \psi_1}
    &
    \braket{\psi_2 \vert W \vert \psi_2}
    \\
  \end{bmatrix}
\end{align}


\begin{solution}
\end{solution}
\end{problem}



\begin{problem}{Vecteurs propres d'un opérateur} %{Marchildon -- 2.13}

La représentation matricielle d'un opérateur $A$ est

\begin{align}
  A = \begin{pmatrix}
    0   &  i / \sqrt{2}  & -i/\sqrt{2}\\
    -i/\sqrt{2}   &   1  & 1\\
     i/\sqrt{2}   &   1  & 1\\
  \end{pmatrix}
\end{align}

\subproblem
Trouvez les valeurs propres $\lambda_i$
  et les vecteurs propres orthonormaux $\ket{\phi_i}$ de A.

\subproblem
À l'aide de la correspondance ket $\leftrightarrow$ matrice-colonne
  et bra $\leftrightarrow$ matrice-vecteur,
  vérifiez que
\begin{align}
  I = \sum_i \ket{\phi_i} \bra{\phi_i}
\end{align}
et
\begin{align}
  A = \sum_i \lambda_i \ket{\phi_i} \bra{\phi_i}
\end{align}

\begin{solution}

\subproblem
On trouve les valeurs propres en solutionnant
\begin{align}
  0 = \begin{vmatrix}
    -\lambda   &  i / \sqrt{2}  & -i/\sqrt{2}\\
    -i/\sqrt{2}   &   1-\lambda  & 1\\
     i/\sqrt{2}   &   1  & 1-\lambda\\
  \end{vmatrix}
\end{align}
Donc
\begin{align}
  0 &=
  -\lambda \big[ (1-\lambda)^2 - 1 \big]
  - \frac{i}{\sqrt{2}}
    \bigg[
      -\frac{i}{\sqrt{2}} (1-\lambda) - \frac{i}{\sqrt{2}}
    \bigg]
  - \frac{i}{\sqrt{2}}
    \bigg[
      - \frac{i}{\sqrt{2}}
      -\frac{i}{\sqrt{2}} (1-\lambda)
    \bigg]\\
  & =
  \lambda (  2\lambda - \lambda^2)
  - \big[ 2 - \lambda \big]\\
  & =
  (2 - \lambda) (\lambda^2 - 1)\\
  & =
  -(\lambda - 2) (\lambda - 1) (\lambda + 1)
\end{align}

Les valeurs propres sont donc -1, 1 et 2.

On trouve les vecteurs propres $(x,y,z)$
  en solutionnant
\begin{align}
  \begin{pmatrix}
    -\lambda   &  i / \sqrt{2}  & -i/\sqrt{2}\\
    -i/\sqrt{2}   &   1-\lambda  & 1\\
     i/\sqrt{2}   &   1  & 1-\lambda\\
  \end{pmatrix}
  \begin{pmatrix}
    x\\
    y\\
    z
  \end{pmatrix}
  = 0\\[3mm]
  \begin{pmatrix}
    -\lambda   &  i / \sqrt{2}  & -i/\sqrt{2}\\
    -i/\sqrt{2}   &   1-\lambda  & 1\\
     0            &   2-\lambda  & 2-\lambda\\
  \end{pmatrix}
  \begin{pmatrix}
    x\\
    y\\
    z
  \end{pmatrix}
  = 0\\[3mm]
  \begin{pmatrix}
    -\lambda   &  i / \sqrt{2}  & -i/\sqrt{2}\\
     0            &   1-\lambda + \tfrac{1}{2\lambda}  & 1 - \tfrac{1}{2\lambda}\\
     0            &   2-\lambda  & 2-\lambda\\
  \end{pmatrix}
  \begin{pmatrix}
    x\\
    y\\
    z
  \end{pmatrix}
  = 0\\[3mm]
  \begin{pmatrix}
     \lambda \sqrt{2} i   &  1 & - 1\\
     0            &   2\lambda-2\lambda^2 + 1 & 2\lambda - 1\\
     0            &   2-\lambda  & 2-\lambda\\
  \end{pmatrix}
  \begin{pmatrix}
    x\\
    y\\
    z
  \end{pmatrix}
  = 0\\[3mm]
\end{align}

Et on untilisera le fait que
  $\vert x \vert^2 + \vert y \vert^2 +\vert z \vert^2=1$.



Pour $\lambda = -1$, on a
\begin{align}
  \begin{pmatrix}
     - \sqrt{2} i   &  1 & - 1\\
     0            &  -3          &-3\\
     0            &   3          & 3\\
  \end{pmatrix}
  \begin{pmatrix}
    x\\
    y\\
    z
  \end{pmatrix}
  = 0\\[3mm]
\end{align}

On a donc $z=-y$ et $y=x i/\sqrt{2}$.
Le premier vecteur propre est donc de la forme
\begin{align}
  \ket{\phi_1} = 
  x
  \begin{pmatrix}
    1\\
    i/\sqrt{2}\\
    -i/\sqrt{2}
  \end{pmatrix}
\end{align}
Et avec la condition de normalisation, on trouve
\begin{align}
  \ket{\phi_1} = 
  \begin{pmatrix}
    1/\sqrt{2}\\
    i/2\\
    -i/2
  \end{pmatrix}
\end{align}



Pour $\lambda = 1$, on a
\begin{align}
  \begin{pmatrix}
     \sqrt{2} i   &  1 & - 1\\
     0            &   1          & 1\\
     0            &   1          & 1\\
  \end{pmatrix}
  \begin{pmatrix}
    x\\
    y\\
    z
  \end{pmatrix}
  = 0\\[3mm]
\end{align}
On a donc $z=-y$ et $y=-x i/\sqrt{2}$.
Le second vecteur propre est donc de la forme
\begin{align}
  \ket{\phi_2} = 
  x
  \begin{pmatrix}
    1\\
    -i/\sqrt{2}\\
    i/\sqrt{2}
  \end{pmatrix}
\end{align}
Et avec la condition de normalisation, on trouve
\begin{align}
  \ket{\phi_2} = 
  \begin{pmatrix}
    1/\sqrt{2}\\
    -i/2\\
     i/2
  \end{pmatrix}
\end{align}

Pour $\lambda = 2$, on a
\begin{align}
  \begin{pmatrix}
     2\sqrt{2} i   &  1 & - 1\\
     0            &  -3          & 3\\
     0            &   0          & 0\\
  \end{pmatrix}
  \begin{pmatrix}
    x\\
    y\\
    z
  \end{pmatrix}
  = 0\\[3mm]
\end{align}

On a donc $z=y$ et $x=0$.
Donc,
\begin{align}
  \ket{\phi_3} = 
  y
  \begin{pmatrix}
     0\\
     1\\
     1
  \end{pmatrix}
\end{align}
Et avec la condition de normalisation, on trouve
\begin{align}
  \ket{\phi_3} = 
  \begin{pmatrix}
     0\\
     1/\sqrt{2}\\
     1/\sqrt{2}
  \end{pmatrix}
\end{align}

\subproblem

Avec les vecteurs propres qu'on a trouvés,
  on peut former les matrices
\begin{align}
  \ket{\phi_1}\bra{\phi_1}
  =
  \begin{pmatrix}
     1/2   & -i/2\sqrt{2}  & i/2\sqrt{2}\\
     i/2\sqrt{2}   & 1/4  & -1/4\\
     -i/2\sqrt{2}  & -1/4  & 1/4 \\
  \end{pmatrix}
\end{align}

\begin{align}
  \ket{\phi_2}\bra{\phi_2}
  =
  \begin{pmatrix}
     1/2   & i/2\sqrt{2}  & -i/2\sqrt{2}\\
     -i/2\sqrt{2}   & 1/4  & -1/4\\
     i/2\sqrt{2}  & -1/4  & 1/4
  \end{pmatrix}
\end{align}

\begin{align}
  \ket{\phi_3}\bra{\phi_3}
  =
  \begin{pmatrix}
      0 & 0   & 0\\
      0 & 1/2 & 1/2\\
      0 & 1/2 & 1/2
  \end{pmatrix}
\end{align}

On a donc bien
\begin{align}
  \sum_i \ket{\phi_i} \bra{\phi_i}
  =
  \begin{pmatrix}
      1 & 0   & 0\\
      0 & 1   & 0\\
      0 & 0   & 1
  \end{pmatrix}
  =
  I
\end{align}
et
\begin{align}
  \sum_i \lambda_i \ket{\phi_i} \bra{\phi_i}
  =
  \begin{pmatrix}
    0   &  i / \sqrt{2}  & -i/\sqrt{2}\\
    -i/\sqrt{2}   &   1  & 1\\
     i/\sqrt{2}   &   1  & 1\\
  \end{pmatrix}
  =
  A
\end{align}


\end{solution}

\end{problem}





\begin{problem}{Mesure d'un opérateur} %{Marchildon 3.1}

L'opérateur $A$ du problème précédent
  est donné sous cette forme matricielle
  dans une certaine base $\ket{u_i}$.
%
Supposons qu'à un certain moment,
  une particule se trouve dans l'état $\ket{\psi}=\ket{u_3}$.
%

\subproblem
Quels sont les résultats possibles d'une mesure de $A$
  et quelles sont les probabilités correspondantes?
  
\subproblem
Quelle est la valeur moyenne de $A$ pour l'état $\ket{\psi}$
  et quelle est la déviation standard $\Delta A$ dans cet état?

\begin{solution}

\subproblem
Les résultats d'une mesure de $A$ sont les valeurs propres $\lambda_i$
  avec probabilités $\vert \braket{\phi_i \vert \psi} \vert^2$.
%
Pour l'état
\begin{align}
  \ket{\psi} = \ket{u_3} =
  \begin{pmatrix}
  0\\
  0\\
  1
  \end{pmatrix}
\end{align}

La probabilité de mesurer $\lambda_1 = -1$ est
\begin{align}
  \vert \braket{\phi_1 \vert u_3} \vert^2
  = \vert -i/2 \vert^2
  = 1/4
\end{align}

La probabilité de mesurer $\lambda_2 = 1$ est
\begin{align}
  \vert \braket{\phi_2 \vert u_3} \vert^2
  = \vert i/2 \vert^2
  = 1/4
\end{align}

La probabilité de mesurer $\lambda_3 = 2$ est
\begin{align}
  \vert \braket{\phi_3 \vert u_3} \vert^2
  = \vert 1/\sqrt{2} \vert^2
  = 1/2
\end{align}

\subproblem
La valeur moyenne de $A$ dans l'état $\ket{\psi}=\ket{u_3}$ 
  est simplement
  $\braket{\psi \vert A \vert \psi} = \braket{u_3 \vert A \vert u_3} = A_{33} = 1$.

\end{solution}
\end{problem}







%\begin{problem}{Marchildon 5.6}
%
%Montrez que, pour des fonctions $\phi(x)$ et $\psi(x)$
%  qui vont vers 0 lorsque $\vert x \vert \rightarrow \infty$,
%\begin{align}
%  \braket{\psi \vert P \vert \phi}
%  =
%  \braket{\phi \vert P \vert \psi}^*
%\end{align}
%Ce qui démontre l'hermiticité de $P$,
%  l'opérateur d'impulsion.
%
%\begin{solution}
%
%On a
%\begin{align}
%  \braket{\psi \vert P \vert \phi}
%  =
%  - i\hbar
%  \int_{-\infty}^{\infty}
%  dx
%  \psi^*(x)
%  \frac{\partial}{\partial x}
%  \phi(x)
%\end{align}
%En intégrant par partie, on a
%\begin{align}
%  \braket{\psi \vert P \vert \phi}
%  &=
%  - i\hbar
%  \psi^*(x)
%  \phi(x)
%  \bigg\vert_{-\infty}^{\infty}
%  + i\hbar
%  \int_{-\infty}^{\infty}
%  dx
%  \phi(x)
%  \frac{\partial}{\partial x}
%  \psi^*(x)\\
%  &=
%  i\hbar
%  \int_{-\infty}^{\infty}
%  dx
%  \phi(x)
%  \frac{\partial}{\partial x}
%  \psi^*(x)\\
%  &=
%  \braket{\phi \vert P \vert \psi}^*
%\end{align}
%
%\end{solution}
%\end{problem}




\end{document}
