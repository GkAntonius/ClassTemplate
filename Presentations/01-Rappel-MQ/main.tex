\documentclass[xcolor=svgnames,t,aspectratio=169,handout]{uqtrcours}
\graphicspath{{img/}}

\usepackage{mylayers}
\usepackage{mysymbols}

% =========================================================================== %
% Info
% =========================================================================== %

\title[]{Rappel des notions élémentaires en mécanique quantique}
\subtitle{PMO1008 -- Mécanique quantique II}
\author[]{Gabriel Antonius}
\date[]{Cours 1}


% =========================================================================== %
% Slides
% =========================================================================== %


\begin{document}

\begin{frame}
  \titlepage
  \thispagestyle{empty}
\end{frame}


\begin{frame}
  \tableofcontents
\end{frame}

\section{Fonction d'onde et opérateurs}
\frame{\sectionpage}


\begin{frame}{Fonction d'onde}

La fonction d'onde $\psi(\vec r)$
  est une fonction réelle ou complexe
  qui encode toute l'information
  sur une particule.

\medskip

La densité de probabilité d'observer une particule à la position $\vec r$
  est
\begin{align}
  \vert \psi(\vec r) \vert^2 = \psi^*(\vec r) \psi(\vec r)
\end{align}

La fonction d'onde doit donc être normalisée:
\begin{equation}
  \int \vert  \psi(\vec r) \vert^2 d^3\vec r = 1
\end{equation}

Aussi, la fonction d'onde doit être continue et différenciable.


\end{frame}



\begin{frame}{Fonction d'onde}

Par exemple, en 1D, si $\psi(x)$ est la fonction d'onde d'une particule,
  alors la probabilité d'observer la particule dans l'intervalle $a \le x \le b$ est
\begin{align}
  \int_a^b \vert \psi(x) \vert^2 dx
\end{align}

\medskip

En 3D, %si $\psi(\vec r)$ est la fonction d'onde d'une particule,
  la probabilité de trouver particule dans le volume
  ($x_1 \le x \le x_2$;
   $y_1 \le y \le y_2$;
   $z_1 \le z \le z_2$)
  est
\begin{align}
  \int_{x_1}^{x_2}
  \int_{y_1}^{y_2}
  \int_{z_1}^{z_2}
  \vert \psi(\vec r) \vert^2 dx dy dz
\end{align}




\end{frame}


%\begin{frame}{Fonction d'onde}
%
%Une fonction d'onde qui décrit $N$ particules s'écrit %$\psi(\vec r_1, \vec r_2,\dots, \vec r_N)$.
%\begin{align}
%  \psi(\vec r_1, \vec r_2,\dots, \vec r_N)
%\end{align}
%
%\medskip
%La norme au carré
%\begin{align}
%  \vert \psi(\vec r_1, \vec r_2,\dots, \vec r_N) \vert^2
%\end{align}
%correspond à la densité de
%  probabilité d'observer en même temps
%  la particule 1 au points $\vec r_1$,
%  la particule 2 au points $\vec r_2$,
%  etc.
%
%\end{frame}



\begin{frame}{Opérateurs}

Un opérateur est un objet qui peut modifier une fonction d'onde.
%

\medskip

Si $\psi(\vec r)$ est une fonction d'onde, et $A$ est un opérateur,
  alors on peut former une nouvelle fonction en faisant
  agir A sur $\psi$ selon
\begin{align}
  \phi(\vec r) = A \psi(\vec r)
\end{align}

\medskip

Remarque: la fonction $\phi(\vec r)$ n'est pas nécessairement
  une fonction d'onde normalisée.

\end{frame}


\begin{frame}{Opérateurs}


La valeur moyenne d'un opérateur pour la particule décrite par la fonction
  d'onde $\psi(r)$ est
\begin{align}
  \braket{A} = \int \psi^*(\vec r) A \psi(\vec r) d^3 \vec r
\end{align}


Par exemple, si une particule se trouve dans un potentiel $V(\vec r)$,
  ce potentiel est un opérateur, et la valeur moyenne du potentiel
  pour l'état $\psi(r)$ est
\begin{align}
  \braket{V} = \int \psi^*(\vec r) V(\vec r) \psi(\vec r) d^3 \vec r
\end{align}

%Les observables sont des quantités réelles
%  que l'on pourrait mesurer, et sont décrites par des opérateurs hermitien.

\end{frame}



\begin{frame}{Opérateur de position}

En 1 dimension, 
  l'opérateur de position, noté $X$,
  agit sur une fonction d'onde
  en la multipliant par la position
%
\begin{align}
  X \psi(x) = x \psi(x) 
\end{align}

La valeur moyenne de l'opérateur de position est
\begin{align}
  \braket{X}
  %=
  %\braket{\psi \vert X \vert \psi}
  =
  \int_{-\infty}^{\infty}
  \psi^*(x)
  x
  \psi(x)
  dx
  =
  \int_{-\infty}^{\infty}
  x
  \vert \psi(x) \vert^2
  dx
\end{align}

\end{frame}



\begin{frame}{Opérateur d'impulsion}

Dans l'espace des positions,
  l'opérateur d'impulsion $P$ en 1D
  est représenté par
\begin{align}
  P  = -i\hbar \frac{\partial}{\partial x}
\end{align}
%
Lorsqu'on fait agir l'opérateur d'impulsion sur une fonction d'onde, on a
\begin{align}
  %\braket{x \vert P \vert \psi}
  P \psi(x)
  =
  -i\hbar \frac{\partial \psi(x)}{\partial x}
\end{align}

On peut calculer la valeur moyenne de l'impulsion
  pour la fonction d'onde $\psi(x)$ avec
\begin{align}
  \braket{P}
  =
  -i\hbar
  \int_{-\infty}^{\infty}
  \psi^*(x)
  \frac{\partial}{\partial x}
  \psi(x)
  dx
\end{align}

La constante de Plank réduite $\hbar$
  possède des unités de positions $\times$ impulsion.

%En général, il existe une correspondance
%  entre les observables et les opérateurs différentiels.

\end{frame}


\begin{frame}{Opérateurs de position et d'impulsion en 3D}

En 3D, l'opérateur de position est noté
\begin{align}
  \vec R = \vec X + \vec Y + \vec Z
\end{align}
Un autre notation possible pour l'opérateur de position est
\begin{align}
  \vec X
  = \vec X_x + \vec X_y + \vec X_z
\end{align}

L'opérateur d'impulsion est
\begin{align}
  \vec P = \vec P_x + \vec P_y + \vec P_z
\end{align}

L'opérateur d'impulsion dans l'espace des positions est représenté par
\begin{align}
  \vec P 
  =
  -i\hbar \nabla
  =
  -i \hbar \left(
    \hat x \frac{\partial}{\partial x}
  + \hat y \frac{\partial}{\partial y}
  + \hat z \frac{\partial}{\partial z}
  \right)
\end{align}
%
%Ceci signifie que
%\begin{align}
%  \braket{\vec r \vert \vec P \vert \psi} = -i\hbar \nabla \psi(\vec r)
%\end{align}

\end{frame}



\begin{frame}{Hamiltonien}

L'Hamiltonien est l'opérateur d'énergie
\begin{equation}
  H = \frac{\vec P^2}{2m} + V(\vec r)
\end{equation}
En 3D, l'opérateur d'impulsion au carré est
\begin{align}
  \vec P^2
  =
  -\hbar^2 \nabla^2
  =
  -\hbar^2 \bigg[
    \frac{\partial^2}{\partial x^2}
  + \frac{\partial^2}{\partial y^2}
  + \frac{\partial^2}{\partial z^2}
  \bigg]
\end{align}

L'énergie moyenne d'une particule en 1D est
\begin{align}
  \braket{H}
  =
  \int \psi^*(x) H \psi(x) dx
  =
  \int
  \psi^*(x)
  \bigg[
    -\frac{\hbar^2}{2m} \frac{\partial^2}{\partial x^2} + V(x)
  \bigg]
  \psi(x)
  dx
\end{align}

\end{frame}




\begin{frame}{Relations de commutation}

En général, l'action de deux opérateurs $A$ et $B$
  sur une fonction d'onde est non-commutative, c'est à dire que
\begin{align}
  A B \psi(\vec r) \neq B A \psi(\vec r)
\end{align}

Le \textbf{commutateur} entre deux opérateurs $A$ et $B$
  est définit comme
\begin{align}
  [A,B] = AB - BA
\end{align}

On dit que $A$ et $B$ commutent si $[A,B]=0$.

\medskip

Dans ce cas, on peut inverser l'ordre de ces deux opérateurs:
\begin{align}
  A B \psi(\vec r) = B A \psi(\vec r)
  \quad \text{ssi} \qquad
  [A,B]=0
\end{align}



%
%\medskip
%Deux opérateurs auront les mêmes fonctions propres
%  si et seulement si ils commutent.
%On aura par exemple
%\begin{align}
%  A \ket{\phi_i} = a_i \ket{\phi_i}
%  \qquad ; \qquad
%  B \ket{\phi_i} = b_i \ket{\phi_i}
%\end{align}
%
%\medskip
%Un ensemble d'opérateurs $(A, B, C, \dots)$ qui commutent est \textbf{complet}
%  si un ensemble de valeurs propres $(a_i, b_i, c_i, \dots)$
%  détermine de façon unique l'état $\ket{\phi_i}$
%  auquel ces valeurs propres correspondent.

\end{frame}



\begin{frame}{Commutation des opérateurs de position et d'impulsion}

En 1D, les opérateurs de position et d'impulsion ne commutent pas,
  et on a
\begin{align}
  [X, P] = i \hbar
\end{align}

\medskip

En 3D, pour les 3 compostantes $X_i$ et $P_i$  ($i=1,2,3$), on a
\begin{align}
  [X_i, X_j] &= 0\\
  [P_i, P_j] &= 0\\
  [X_i, P_j] &= i\hbar \delta_{ij}
\end{align}

\end{frame}






\begin{frame}{Observables}


Un \textbf{observable} est une quantité physique que l'on peut mesurer.
%
Par exemple:
\begin{itemize}
  \item La position
  \item L'impulsion
  \item Le moment cinétique
  \item L'énergie
\end{itemize}

\medskip

Un observable est représenté par un opérateur
  des valeurs propres réelles (un opérateur hermitien).

\end{frame}





\section{Équation de Schrödinger}
\frame{\sectionpage}




\begin{frame}{Équation de Schödinger}


L'équation de Schrödinger nous donne l'évolution temporelle 
  de la fonction d'onde.
\begin{align}
  i \hbar \frac{\partial}{\partial t}
  \psi(\vec r, t)
  =
  H \psi(\vec r, t)
\end{align}

%Une fonction d'onde correspond à un \textbf{état propre}
Si une fonction d'onde correspond à un état propre de l'hamiltonien
\begin{align}
  H \psi_{j}(\vec r, t)
  =
  E_{j} \psi_{j}(\vec r, t)
\end{align}

Alors l'évolution temporelle de cette fonction d'onde
  sera donnée par
\begin{align}
  \psi_{j}(\vec r, t) = e^{-i E_{j} t / \hbar} \psi_{j}(\vec r, 0)
\end{align}

\end{frame}



\begin{frame}{Équation de Schödinger}

L'équation de Schrödinger est donc une équation
  aux valeurs propres.


%\begin{align}
%  H \psi_j(\vec r, t)
%  =
%  E_{j} \psi_{j}(\vec r, t)
%\end{align}

\medskip

Au temps $t=0$, la fonction d'onde $\psi_j(r) = \psi_j(\vec r, 0)$
  obéit à l'équation différentielle
\begin{align}
    \left[
    -\frac{\hbar^2}{2m} \nabla^2 + V(\vec r) \right] \psi_j(\vec r)
    = E_j \psi_j(\vec r)
\end{align}

L'indice $j$ est un \textbf{nombre quantique}
  et indique que la fonction d'onde $\psi_{j}$
  est la $j^e$ solution possible de l'équation de Schrödinger,
  et l'énergie $E_{j}$ est la $j^e$ valeur propre de l'Hamiltonien.

\medskip
Les solutions de $H$ peuvent aussi être continues
  et être décrites par un nombre quantique réel,
  par exemple, pour une particule libre
\begin{align}
  E(\vec p) = \frac{\vec p^2}{2m}
\end{align}

\end{frame}




\begin{frame}{États dégénérés}

%\begin{align}
%  H \ket{\phi_j} = E_j \ket{\phi_j}
%\end{align}

Lorsque deux états propres différents $\phi_i$ et $\phi_j$
  possèdent la même valeur propre $E_i=E_j$,
  on dit que ces états sont \textbf{dégénérés}.

\medskip
La dégénérescence d'un état $g_j$ est le nombre d'états qui possèdent
  la même énergie $E_j$.
%\medskip
%Quand les états sont dégénérés,
On introduit alors un nombre quantique
  additionnel $\alpha$ pour distinguer les différents états dégénérés
\begin{align}
  H \phi_{j\alpha} = E_j \phi_{j\alpha}
  \quad ; \quad \alpha = 1, \dots, g_j
\end{align}
%On dit que les indices $j$ et $\alpha$ sont des \textbf{nombres quantiques}.
%L'indice $j$ est un \textbf{nombre quantique}


\end{frame}




\section{Notation de Dirac}
\frame{\sectionpage}







\begin{frame}{Notation de Dirac}

La notation de Dirac traite les fonctions d'ondes comme des
  vecteurs et les opérateurs comme des matrices.

\medskip

On note $\ket{\psi}$ le \textit{ket} de la fonction d'onde,
  et $\bra{\psi}$ le \textit{bra} de la fonction d'onde,
  avec la correspondance
\begin{align}
  \ket{\psi} & \longrightarrow \psi(\vec r) \nonumber\\
  \bra{\psi} & \longrightarrow \psi^*(\vec r)
\end{align}


Le \textit{braket} $\braket{\phi \vert \psi}$
  correspond à un produit scalaire,
  ou encore à la projection de $\psi$ sur $\phi$:
%
\begin{align}
  \braket{\phi \vert \psi}
  =
  %\int_{-\infty}^{\infty}
  \int
  \phi^*(\vec r)
  \psi(\vec r)
  d^3 \vec r
\end{align}

\end{frame}


\begin{frame}{Notation de Dirac}

Par exemple, pour deux fonctions d'onde en 1D, on a
\begin{align}
  \braket{\phi \vert \psi}
  =
  \int_{-\infty}^{\infty}
  dx
  \phi^*(x)
  \psi(x)
\end{align}
%
Pour un fonction d'onde à une particule en 1D, on a
\begin{align}
  \braket{\psi \vert \psi}
  =
  \int_{-\infty}^{\infty}
  dx \vert \psi(x) \vert^2
  = 1
\end{align}
%
Et de façon générale,
  la fonction d'onde d'une particule
  est toujours normalisée,
  de sorte que
  %$\braket{\psi \vert \psi}=1$.
\begin{align}
  \braket{\psi \vert \psi} = 1
\end{align}

\end{frame}





\begin{frame}{Notation de Dirac}

La notation de Dirac permet d'exprimer la valeur moyenne d'un opérateur comme
\begin{align}
  \braket{A}
  = \braket{\psi \vert A \vert \psi}
  = \int \psi^*(\vec r) A \psi(\vec r) d^3 \vec r
\end{align}

Remarque: si l'application d'un opérateur $A$
  sur la fonction d'onde $\ket{\psi}$ est
\begin{align}
  \ket{\phi} = A \ket{\psi}
\end{align}
alors
\begin{align}
  \bra{\phi} = \bra{\psi} A^{\dagger}
\end{align}
où $A^{\dagger}$ (prononcé A \textit{dagger})
  est le conjugué hermitien de $A$
  (le conjugué complexe de la matrice transposée).

\end{frame}



%\begin{frame}{Notation de Dirac}
%
%La projection de $\ket{\psi}$ sur l'espace des positions
%  est
%\begin{align}
%  \psi(\vec r) = \braket{\vec r \vert \psi}
%\end{align}
%%Tandis que
%\begin{align}
%  \psi^*(\vec r) = \braket{\psi \vert \vec r}
%\end{align}
%
%\end{frame}


%\begin{frame}{Opérateur de position}
%
%En 1 dimension, 
%  l'opérateur de position, noté $X$,
%  agit sur une fonction d'onde
%  en la multipliant par la position
%%
%\begin{align}
%  X \psi(x) = x \psi(x) 
%\end{align}
%%
%Pour être vraiment rigoureux,
%  il faudrait plutôt écrire
%\begin{align}
%  \bra{x} X \ket{\psi}
%  = x \braket{x \vert \psi}
%  = x \psi(x) 
%\end{align}
%
%
%La valeur moyenne de l'opérateur de position est
%\begin{align}
%  \braket{X}
%  =
%  \braket{\psi \vert X \vert \psi}
%  =
%  \int_{-\infty}^{\infty}
%  dx
%  \psi^*(x)
%  x
%  \psi(x)
%  =
%  \int_{-\infty}^{\infty}
%  dx
%  \vert \psi(x) \vert^2
%  x
%\end{align}
%
%\end{frame}
%
%
%
%
%
%\begin{frame}{Opérateur d'impulsion}
%
%Dans l'espace des positions,
%  l'opérateur d'impulsion $P$
%  est représenté par
%\begin{align}
%  P  \longrightarrow -i\hbar \frac{\partial}{\partial x}
%\end{align}
%%
%Ceci signifie que
%\begin{align}
%  \braket{x \vert P \vert \psi}
%  =
%  -i\hbar \frac{\partial \psi(x)}{\partial x}
%\end{align}
%
%
%La valeur moyenne de l'impulsion
%  %pour une particule décrite par la fonction d'onde $\psi(x)$
%  sera
%\begin{align}
%  \braket{P}
%  =
%  -\int_{-\infty}^{\infty}
%  dx
%  \psi^*(x)
%  i\hbar \frac{\partial}{\partial x}
%  \psi(x)
%\end{align}
%
%En général, il existe une correspondance
%  entre les observables et les opérateurs différentiels.
%
%\end{frame}
%
%
%
%\begin{frame}{Opérateurs de position et d'impulsion en 3D}
%
%En 3D, l'opérateur de position est noté
%\begin{align}
%  \vec R = \vec X + \vec Y + \vec Z
%\end{align}
%ou encore
%\begin{align}
%  \vec X
%  = \vec X_x + \vec X_y + \vec X_z
%\end{align}
%
%L'opérateur d'impulsion est
%\begin{align}
%  \vec P = \vec P_x + \vec P_y + \vec P_z
%\end{align}
%
%L'opérateur d'impulsion dans l'espace des positions est représenté par
%\begin{align}
%  P  \longrightarrow -i\hbar \nabla
%\end{align}
%%
%Ceci signifie que
%\begin{align}
%  \braket{\vec r \vert \vec P \vert \psi} = -i\hbar \nabla \psi(\vec r)
%\end{align}
%
%\end{frame}








%\begin{frame}{Opérateurs et observables}
%
%%On peut noter $\hat x$ l'opérateur de position selon
%%  l'axe des positions $x$.
%%
%
%Pour une particule en 1 dimension
%  décrite par la fonction d'onde $\psi(x)$,
%  la valeur moyenne de l'opérateur de position sera
%\begin{align}
%  \braket{X}
%  =
%  \int_{-\infty}^{\infty}
%  dx
%  \psi^*(x)
%  x
%  \psi(x)
%  =
%  \int_{-\infty}^{\infty}
%  dx
%  \vert \psi(x) \vert^2
%  x
%\end{align}
%
%\medskip
%
%Il existe une correspondance %un-pour-un
%  entre les observables
%  et les opérateurs différentiels.
%%
%
%\medskip
%Par exemple, en 1 dimenion, on a pour l'impulsion
%\begin{align}
%  P  \longrightarrow -i\hbar \frac{\partial}{\partial x}
%\end{align}
%
%La valeur moyenne de l'impulsion
%  %pour une particule décrite par la fonction d'onde $\psi(x)$
%  sera
%\begin{align}
%  \braket{P}
%  =
%  -\int_{-\infty}^{\infty}
%  dx
%  \psi^*(x)
%  i\hbar \frac{\partial}{\partial x}
%  \psi(x)
%\end{align}
%
%
%\end{frame}



%\begin{frame}{Opérateurs et observables}
%
%Il existe une correspondance %un-pour-un
%  entre les observables
%  et les opérateurs différentiels.
%%
%
%\medskip
%Par exemple, en 1 dimenion, on a pour la position
%\begin{align}
%  p  \longrightarrow i\hbar \frac{\partial}{\partial x}
%\end{align}
%
%La valeur moyenne de l'impulsion pour une particule
%  décrite par la fonction d'onde $\psi(x)$ sera
%\begin{align}
%  \braket{p}
%  =
%  \int_{-\infty}^{\infty}
%  dx
%  \psi^*(x)
%  i\hbar \frac{\partial}{\partial x}
%  \psi(x)
%\end{align}
%
%\end{frame}



\section{États propres et valeurs propres des opérateurs}
\frame{\sectionpage}


%\begin{frame}{États propres de l'Hamiltonien}
%
%Les états propres de l'Hamiltonien sont les
%  solutions de l'équation de Schrödinger
%\begin{align}
%  H \ket{\phi_j} = E_j \ket{\phi_j}
%\end{align}
%La valeur propre $E_j$ correspond à l'énergie d'un système dans l'état
%  $\ket{\phi_j}$.
%
%%L'indice $j$ est un \textbf{nombre quantique}
%%  et indique que la fonction d'onde $\psi_{j}$
%%  est la $j^e$ solution possible de l'équation de Schrödinger,
%%  et l'énergie $E_{j}$ est la $j^e$ valeur propre de l'Hamiltonien.
%
%%\medskip
%
%%Les états propres de l'hamiltonien peuvent être discrets
%%  ou continus.
%  
%
%\end{frame}



\begin{frame}{Valeurs propres et fonctions propres d'un opérateur}

Un opérateur possède des
  valeurs propres $a_i$
  et des états propres $\ket{\phi_i}$
  tel que
\begin{align}
  A \ket{\phi_i} = a_i \ket{\phi_i}
\end{align}

\medskip

Les états propres d'un opérateur sont orthonormés:
\begin{align}
  \braket{\phi_i \vert \phi_j}
  =
  \int \phi_i^*(\vec r) \phi_j (\vec r) d^3 \vec r
  =
  \delta_{ij}
\end{align}

\medskip

On peut exprimer un opérateur
  en termes de ses fonctions propres
  et de ses valeurs propres comme
\begin{align}
  A = \sum_i \ket{\phi_i} a_i \bra{\phi_i}
\end{align}

Si toutes les valeurs propres d'un opérateur sont réelles,
  alors on dit que c'est un opérateur \textbf{hermitien}.

L'ensemble des états propres d'un opérateur forment une base complète.

\end{frame}


\begin{frame}{Opérateurs qui commutent}

\textbf{Théorème:}
Si deux opérateurs $A$ et $B$ commutent ($AB=BA$), alors ils possèdent les mêmes états propres.

\medskip
\textbf{Preuve:} Si $\ket{\phi_i}$ est un état propre de $A$ avec valeur propre $a_i$, alors
\begin{align*}
  AB \ket{\phi_i} &= BA \ket{\phi_i}
  \\
  AB \ket{\phi_i} &=  a_i B \ket{\phi_i}
  \\
  A \ket{\tilde \phi_i} &=  a_i \ket{ \tilde \phi_i}
\end{align*}
Donc, l'état $\ket{\tilde \phi_i} = B \ket{\phi_i}$
  est aussi un état propre de $A$ avec valeur propre $a_i$.

\medskip
Cet état est forcément proportionnel à $\ket{\phi_i}$.
Si on nomme cette constante de proportionalité $b_i$,
  alors on a
\begin{align}
  \ket{\tilde \phi_i} = B \ket{\phi_i} = b_i \ket{\phi_i}
\end{align}
Donc, $\ket{\phi_i}$ est aussi un état propre de $B$,
  avec valeur propre $b_i$.


\end{frame}





\begin{frame}{États propres de l'opérateur de position}

On note $\ket{\vec r}$ un état propres de l'opérateur de position
  tel que
\begin{align}
  \vec X \ket{\vec r} = \vec r \ket{\vec r}
\end{align}

Lorsqu'on écrit $\psi(\vec r)$, on parle en fait de la projection
  de $\ket{\psi}$ sur l'état $\ket{\vec r}$
\begin{align}
  \psi(\vec r) = \braket{\vec r \vert \psi}
\end{align}

Les états $\ket{\vec r}$ sont orthonormés, tel que
\begin{align}
  \braket{\vec r \vert \vec r'} = \delta^3(\vec r - \vec r')
\end{align}
Ils décrivent une fonction d'onde infiniement étroite
  autour de la position $\vec r$.

\end{frame}



\begin{frame}{États propres de l'opérateur d'impulsion}

On note $\ket{\vec p}$ un état propres de l'opérateur d'impulsion
  tel que
\begin{align}
  \vec P \ket{\vec p} = \vec p \ket{\vec p}
\end{align}

Les états $\ket{\vec p}$ sont orthonormés, tel que
\begin{align}
  \braket{\vec p \vert \vec p'} = \delta^3(\vec p - \vec p')
\end{align}
Ils décrivent une fonction d'onde avec une impulsion bien définie,
  ou encore, une particule libre.
%
La projection de ces états dans l'espace est
%
\begin{align}
  \psi(\vec r) = \braket{\vec r \vert \vec p}
  = e^{i \vec p \cdot \vec r / \hbar}
  %= e^{i \vec k \cdot \vec r}
\end{align}

\end{frame}








%\section{Équation de Schrödinger}
%\frame{\sectionpage}


%\begin{frame}{États propres de l'Hamiltonien}
%
%Les états propres de l'Hamiltonien sont les
%  solutions de l'équation de Schrödinger
%\begin{align}
%  H \ket{\phi_j} = E_j \ket{\phi_j}
%\end{align}
%La valeur propre $E_j$ correspond à l'énergie d'un système dans l'état
%  $\ket{\phi_j}$.
%
%\medskip
%
%%\medskip
%
%Si les états propres de $H$ sont dénombrables
%  ($j=0,1,2,\dots$) avec des énergies distinctes
%  ($E_0 < E_1 < E_2 < \dots$),
%  alors on dit que $H$ possède des solutions discrètes,
%  et l'indice $j$ est un \textbf{nombre quantique} entier.
%
%\medskip
%Les solutions de $H$ peuvent aussi être continues
%  et être décrites par un nombre quantique réel,
%  par exemple, pour une particule libre
%\begin{align}
%  H \ket{\vec p} = E(\vec p) \ket{\vec p}
%\end{align}
%
%%L'indice $j$ est un \textbf{nombre quantique}
%%  et indique que la fonction d'onde $\psi_{j}$
%%  est la $j^e$ solution possible de l'équation de Schrödinger,
%%  et l'énergie $E_{j}$ est la $j^e$ valeur propre de l'Hamiltonien.
%
%%\medskip
%
%%Les états propres de l'hamiltonien peuvent être discrets
%%  ou continus.
%  
%
%\end{frame}


%\begin{frame}{Équation de Schödinger}
%
%L'équation de Schrödinger nous donne l'évolution temporelle 
%  de la fonction d'onde.
%\begin{align}
%  i \hbar \frac{\partial}{\partial t}
%  \psi(\vec r, t)
%  =
%  H \psi(\vec r, t)
%\end{align}
%
%%Une fonction d'onde correspond à un \textbf{état propre}
%Si une fonction d'onde correspond à un état propre de l'hamiltonien
%\begin{align}
%  H \psi_{j}(\vec r, t)
%  =
%  E_{j} \psi_{j}(\vec r, t)
%\end{align}
%%L'indice $j$ est un \textbf{nombre quantique}
%%  et indique que la fonction d'onde $\psi_{j}$
%%  est la $j^e$ solution possible de l'équation de Schrödinger,
%%  et l'énergie $E_{j}$ est la $j^e$ valeur propre de l'Hamiltonien.
%
%
%Alors l'évolution temporelle de cette fonction d'onde
%  sera donnée par
%\begin{align}
%  \psi_{j}(\vec r, t) = e^{-i E_{j} t / \hbar} \psi_{j}(\vec r, 0)
%\end{align}
%
%\end{frame}





\begin{frame}{États propres de l'hamiltonien}

Les états propres de l'Hamiltonien sont
\begin{align}
  H \ket{\phi_j} = E_j \ket{\phi_j}
\end{align}

Si au temps $t=0$, une fonction d'onde est une combinaison linéaire
  de ces états
\begin{align}
  \ket{\psi(0)} = \sum_{j} c_j \ket{\phi_j}
\end{align}

On peut calculer les coefficients avec
\begin{align}
  c_j = \braket{\phi_j \vert \psi(0)}
\end{align}

Au temps $t$, l'évolution temporelle de la fonction d'onde nous donne
\begin{align}
  \ket{\psi(t)} = \sum_{j} c_j e^{-i E_{j} t / \hbar} \ket{\phi_j}
\end{align}

\end{frame}





\section{Mesure d'un observable}
\frame{\sectionpage}



\begin{frame}{Mesure d'un observable}


Soit un observable $A$, qui possède des
  valeurs propres $a_i$ et des
  fonctions propres $\ket{\phi_i}$,
  et soit une particule dont la fonction d'onde est $\ket{\psi}$.

\bigskip
Si l'on mesure $A$, les résultats possibles sont les valeurs propres $a_i$,
  chacune avec une probabilité $\vert \braket{\phi_i \vert \psi} \vert^2$.
%

\bigskip
Immédiatement après avoir mesurée la valeur $a_i$ pour l'observable $A$,
  la fonction d'onde est projetée dans l'état $\ket{\phi_i}$

\end{frame}




\begin{frame}{Mesure d'un observable}

La valeur moyenne, ou valeur attendue, d'un observable $A$ pour l'état $\ket{\psi}$, est
\begin{align}
  \braket{A} =
  \braket{\psi \vert A \vert \psi} = \sum_{i} a_i \vert \braket{\phi_i \vert \psi} \vert^2
\end{align}

\bigskip

La variance d'un observable $(\Delta A)^2$ est définie comme
\begin{align}
  (\Delta A)^2 = \braket{A^2} - \braket{A}^2
\end{align}

Et $\Delta A$ est appellé la déviation standard de $A$.

\end{frame}





\section{Bases complètes de fonctions orthonormales}
%\section{Bases discrètes et continues}
\frame{\sectionpage}




\begin{frame}{Opérateur d'identité}

L'identité $I$ est un opérateur qui ne modifie pas la fonction d'onde

\bigskip

\begin{align}
  I \ket{\psi} = \ket{\psi}
\end{align}

\bigskip

Il existe plusieurs façons de construire l'identité...

\end{frame}




\begin{frame}{Base complète de fonctions orthonormales}

%Pour un espace vectoriel de dimension infinie,
%  le nombre de fonctions de bases est illimité.
%
En général, il existe une infinité d'états propres
  de l'hamiltonien (ou de tout autre opérateur):
\begin{align}
  H \ket{\phi_j} = E_j \ket{\phi_j}
\end{align}

Ces fonctions sont orthonormales:
\begin{align}
  \braket{\phi_i \vert \phi_j} = \delta_{ij}
\end{align}

Elles forment une base complète orthonormale
  avec l'identité
\begin{align}
  I = \sum_{j=1}^{\infty} \ket{\phi_j} \bra{\phi_j}
\end{align}

\end{frame}



\begin{frame}{Décomposition de la fonction d'onde}

Une fonction d'onde peut être décomposée
  dans la base des fonctions propres de l'hamiltonien
\begin{align}
  \ket{\psi} = \sum_{j=1}^{\infty} c_j \ket{\phi_j}
\end{align}

Les coefficients de la fonction d'onde $\ket{\psi}$
  dans cette base sont
\begin{align}
  c_i = \braket{\phi_i \vert \psi}
\end{align}

On peut démontrer ceci à l'aide de l'opérateur d'identité
\begin{align}
  \ket{\psi}
  = I \ket{\psi}
  = \sum_{j=1}^{\infty} \ket{\phi_j}\braket{\phi_j \vert \psi}
  = \sum_{j=1}^{\infty} \ket{\phi_j} c_j
\end{align}

\end{frame}


%
%\begin{frame}{Bases continues}
%
%Une base infinie peut être continue
%  si les fonctions de base sont identifiées
%  par des nombres réels plutôt que des nombres entiers.
%
%\medskip
%
%Par exemple, les états propres de l'opérateur de position $\ket{\vec r}$
%  forment une base complète:
%\begin{align}
%  I = \int d^3\vec r \ket{\vec r} \bra{\vec r}
%\end{align}
%%et orthonormale:
%%\begin{align}
%%  \braket{\vec r \vert \vec r'} = \delta^3(\vec r - \vec r')
%%\end{align}
%
%C'est aussi le cas des états propres de l'opérateur d'impulsion:
%\begin{align}
%  I & = \int d^3\vec p \ket{\vec p} \bra{\vec p}
%\end{align}
%%avec
%%\begin{align}
%%  \braket{\vec p \vert \vec p'} &= \delta^3(\vec p - \vec p')
%%\end{align}
%
%\end{frame}



\begin{frame}{Espace vectoriel}

Si l'on dispose d'un ensemble de $N$ fonctions orthornormales
  $\ket{u_i}$ tel que
\begin{align}
  \braket{u_i \vert u_j} = \delta_{ij}
\end{align}

Ces $N$ fonctions définissent un \textbf{espace vectoriel} à $N$ dimensions,
  et dans cet espace vectoriel, on a l'identité
\vspace{-3mm}
\begin{align}
  I = \sum_{i=1}^N \ket{u_i} \bra{u_i}
\end{align}

Une fonction d'onde $\ket{\psi}$
  fait partie de cet espace vectoriel
  si on peut l'exprimer comme une combinaison linéaire
\vspace{-3mm}
\begin{align}
  \ket{\psi} = \sum_{i=1}^N c_i \ket{u_i}
\end{align}

Les coefficients de la fonction d'onde $\ket{\psi}$
  dans la base de $\ket{u_i}$ sont
\begin{align}
  c_i = \braket{u_i \vert \psi}
\end{align}

\end{frame}




%
%
%
%\begin{frame}{Équation de Schödinger}
%
%L'équation de Schrödinger nous donne l'évolution temporelle 
%  de la fonction d'onde.
%\begin{align}
%  i \hbar \frac{\partial}{\partial t}
%  \psi(\vec r, t)
%  =
%  H \psi(\vec r, t)
%\end{align}
%
%%Une fonction d'onde correspond à un \textbf{état propre}
%Si une fonction d'onde correspond à un état propre de l'hamiltonien
%\begin{align}
%  H \psi_{j}(\vec r, t)
%  =
%  E_{j} \psi_{j}(\vec r, t)
%\end{align}
%%L'indice $j$ est un \textbf{nombre quantique}
%%  et indique que la fonction d'onde $\psi_{j}$
%%  est la $j^e$ solution possible de l'équation de Schrödinger,
%%  et l'énergie $E_{j}$ est la $j^e$ valeur propre de l'Hamiltonien.
%
%
%Alors l'évolution temporelle de cette fonction d'onde
%  sera donnée par
%\begin{align}
%  \psi_{j}(\vec r, t) = e^{-i E_{j} t / \hbar} \psi_{j}(\vec r, 0)
%\end{align}
%
%\end{frame}
%
%
%
%\begin{frame}{Évolution temporelle de la fonction d'onde}
%
%Si on connaît $\ket{\psi(0)}$, la fonction d'onde au temps $t=0$,
%  pour calculer son évolution au temps $t$,
%  il faut commencer par la décomposer
%  dans la base des états propres de l'hamiltonien
%\begin{align}
%  \ket{\psi(0)} = \sum_{j} c_j \ket{\phi_j}
%\end{align}
%avec
%\begin{align}
%  c_j = \braket{\phi_j \vert \psi(0)}
%\end{align}
%
%Au temps $t$, la fonction d'onde deviendra
%\begin{align}
%  \ket{\psi(t)} = \sum_{j} c_j e^{-i E_{j} t / \hbar} \ket{\phi_j}
%\end{align}
%
%
%\end{frame}
























\section{Représentation matricielle des opérateurs}
\frame{\sectionpage}



\begin{frame}{Représentation matricielle d'un opérateur}

Il est toujours possible de représenter un opérateur comme une matrice
  à l'aide d'un ensemble de fonctions de bases $\ket{u_i}$ avec
\begin{align}
  A_{ij} = \braket{u_i \vert A \vert u_j}
\end{align}

Par exemple, dans un espace vectoriel à 3 dimensions
  avec les fonctions de base $\ket{u_1}$, $\ket{u_2}$, $\ket{u_3}$,
  on exprimerait un opérateur comme
\begin{align}
  A =
  \begin{bmatrix}
    A_{11} & A_{12} & A_{13} \\
    A_{21} & A_{22} & A_{23} \\
    A_{31} & A_{32} & A_{33} \\
  \end{bmatrix}
  =
  \begin{bmatrix}
    \braket{u_1 \vert A \vert u_1} & \braket{u_1 \vert A \vert u_2} & \braket{u_1 \vert A \vert u_3} \\
    \braket{u_2 \vert A \vert u_1} & \braket{u_2 \vert A \vert u_2} & \braket{u_2 \vert A \vert u_3} \\
    \braket{u_3 \vert A \vert u_1} & \braket{u_3 \vert A \vert u_2} & \braket{u_3 \vert A \vert u_3} \\
  \end{bmatrix}
\end{align}

Un terme $\braket{u_i \vert A \vert u_j}$
  est appelé un \textbf{élément de matrice}.

\end{frame}


%
%\begin{frame}{Valeurs propres et fonctions propres d'un opérateur}
%
%Un opérateur possède des fonctions propres $\ket{\phi_i}$
%  et des valeurs propres $a_i$ si
%\begin{align}
%  A \ket{\phi_i} = a_i \ket{\phi_i}
%\end{align}
%
%Un opérateur sera \textbf{hermitien} si $A\!=\!A^{\dagger}$,
%  c'est-à-dire
%\begin{align}
%  A_{ij} = A_{ji}^*
%\end{align}
%Un opérateur hermitien possède des valeurs propres réelles.
%
%\medskip
%
%Les \textbf{observables} sont des opérateurs hermitiens.
%
%\medskip
%
%On peut exprimer un opérateur
%  en termes de ses fonctions propres
%  et de ses valeurs propres comme
%\begin{align}
%  A = \sum_i \ket{\phi_i} a_i \bra{\phi_i}
%\end{align}
%
%\end{frame}



\begin{frame}{Valeurs propres et fonctions propres d'un opérateur}

La représentation matricielle des opérateurs permet de trouver
  leurs valeurs propres et fonctions propres.
Par exemple, l'opérateur
\begin{align}
  A =
  \begin{bmatrix}
    0 & 1 & 0 \\
    1 & 0 & 0 \\
    0 & 0 & 2 \\
  \end{bmatrix}
\end{align}
possède les valeurs propres
\begin{align}
  a_1 = -1
  \qquad ; \qquad
  a_2 =  1
  \qquad ; \qquad
  a_3 =  2
\end{align}
et les vecteurs propres
\begin{align}
  \ket{\phi_1} =
  \frac{1}{\sqrt{2}}
  \begin{bmatrix}
    1 \\
    -1 \\
    0 \\
  \end{bmatrix}
  \quad ; \quad
  \ket{\phi_2} =
  \frac{1}{\sqrt{2}}
  \begin{bmatrix}
    1 \\
    1 \\
    0 \\
  \end{bmatrix}
  \quad ; \quad
  \ket{\phi_3} =
  \begin{bmatrix}
    0 \\
    0 \\
    1 \\
  \end{bmatrix}
\end{align}

\end{frame}



\begin{frame}{Valeurs propres et fonctions propres d'un opérateur}

Remarque 1: si on a exprimé un opérateur dans la base
  des $\ket{u_i}$, alors un état comme
\begin{align}
  \ket{\psi} =
  \frac{1}{\sqrt{2}}
  \begin{bmatrix}
    1 \\
    i \\
    0 \\
  \end{bmatrix}
\end{align}
signifie en fait
\begin{align}
  \ket{\psi} = \frac{1}{\sqrt{2}} \ket{u_1} + \frac{i}{\sqrt{2}} \ket{u_2}
\end{align}

Remarque 2: un ket est représenté par un vecteur-colonne,
  et un bra est représenté par un vecteur-ligne, e.g.
\begin{align}
  \bra{\psi}
  \ = \
  \frac{1}{\sqrt{2}}
  \begin{bmatrix}
    1  , &
    -i , &
    0 \\
  \end{bmatrix}
  \ = \
  \frac{1}{\sqrt{2}} \bra{u_1} - \frac{i}{\sqrt{2}} \bra{u_2}
\end{align}

\end{frame}





\end{document}
