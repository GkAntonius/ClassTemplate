\documentclass[12pt,solution=false]{uqtrassignment}
\usepackage[utf8]{inputenc}
\usepackage{amsmath}
\usepackage{braket}
\usepackage{mysymbols}
\usepackage{color}
\usepackage{graphicx}
\usepackage{wrapfig}
\graphicspath{{img/}}

\coursecode{PMO1008}
\semester{Automne 2022}
\authormail{gabriel.antonius@uqtr.ca}
\department{Département de chimie, biochimie et physique}
\setdocumentlabel{Devoir}

\title{Devoir 1}
\author{Gabriel Antonius}


% ============================================================================ %

\begin{document}

\maketitle

\begin{problem}{}
% Cohen-Tannoudji exercice 3.14

Considérez un système physique dont l'espace des états
  est décrit par la base des kets $\ket{u_1}$,  $\ket{u_2}$,  $\ket{u_3}$. 
Dans cette base, l'Hamiltonien, ainsi que deux autres observables $A$ et $B$
  sont donnés par


\begin{align*}
  H =
  \hbar \omega_0
  \begin{pmatrix}
     1 & 0 & 0 \\
     0 & 2 & 0 \\
     0 & 0 & 2 \\
  \end{pmatrix}
  \quad ; \qquad
  A =
  a
  \begin{pmatrix}
     1 & 0 & 0 \\
     0 & 0 & 1 \\
     0 & 1 & 0 \\
  \end{pmatrix}
  \quad ; \qquad
  B =
  b
  \begin{pmatrix}
     0 & 1 & 0 \\
     1 & 0 & 0 \\
     0 & 0 & 1 \\
  \end{pmatrix}
\end{align*}
où les constantes $\omega_0$, $a$, et $b$ sont toutes réelles et positives.

Au temps $t=0$, le système est décrit par la fonction d'onde
\begin{align}
  \ket{\psi(0)} = \frac{1}{\sqrt{2}} \ket{u_1}
                + \frac{1}{2} \ket{u_2}
                + \frac{1}{2} \ket{u_3}
\end{align}


\subproblem
Au temps $t=0$, on mesure l'énergie du système.
Quels résultats peut-on trouver, et avec quelles probabilités?
Donnez, pour le système dans l'état $\ket{\psi(0)}$
  la valeur moyenne $\braket{H}$
  et la déviation standard $\Delta H$.


\subproblem
Plutôt que de mesurer $H$ au temps $t=0$,
  on mesure l'observable $A$.
Quels résultats peut-on trouver et avec quelles probabilités?
Quel est l'état du système immédiatement après la mesure?

\subproblem
Calculez l'état du système $\ket{\psi(t)}$ au temps $t$.

\subproblem
Calculez au temps $t$ la valeur moyenne des observables
  $\braket{A}(t)$ et $\braket{B}(t)$
  et donnez-en une interprétation physique.



\subproblem
Quels résultats peuvent être obtenus si l'on mesure les observables
  $A$ et $B$ au temps $t$?
Quelle est l'interprétation physique?


\begin{solution}

\subproblem
L'Hamiltonien est diagonal dans la base des kets
  $\ket{u_1}$,  $\ket{u_2}$,  $\ket{u_3}$. 
Les valeurs propres correspondantes sont
\begin{align}
  E_1 = \hbar \omega_0
  \qquad ; \qquad
  E_2 = 2 \hbar \omega_0
  \qquad ; \qquad
  E_3 = 2 \hbar \omega_0
\end{align}
Une mesure de l'énergie peut donner les valeurs propres
  $E_1$, $E_2$, $E_3$.
%
La probabilité d'observer l'énergie $E_1=\hbar \omega_0$ est
\begin{align}
  P(\hbar\omega_0) = P(E_1) = \vert \braket{u_1 \vert \psi(0)} \vert^2 = \frac{1}{2}
\end{align}
Puisque $E_2=E_3=2\hbar\omega_0$, la probabilité d'observer ce résultat est
\begin{align}
  P(2\hbar\omega_0) = P(E_2) + P(E_3)
  = \vert \braket{u_2 \vert \psi(0)} \vert^2
  + \vert \braket{u_3 \vert \psi(0)} \vert^2
  = \frac{1}{2}
\end{align}

La valeur moyenne de l'énergie sera
\begin{align}
  \braket{H}
  &= E_1 P(E_1) + E_2 P(E_2) + E_3 P(E_3)
  \\
  &= \hbar\omega_0 P(\hbar\omega_0) + 2 \hbar\omega_0 P(2\hbar\omega_0)
  \\
  &= \hbar\omega_0 \frac{1}{2} + 2 \hbar\omega_0 \frac{1}{2}
  = \frac{3}{2} \hbar \omega_0
\end{align}

La variance de $H$ dans cet état est
\begin{align}
  \Delta H^2 = \braket{H^2} - \braket{H}^2
  &= E_1^2 P(E_1) + E_2^2 P(E_2) + E_3^2 P(E_3) - \braket{H}^2
  \\
  &= (\hbar\omega_0)^2 \frac{1}{2} + (2 \hbar\omega_0)^2 \frac{1}{4} + (2 \hbar\omega_0)^2 \frac{1}{4}
  - \left(\frac{3}{2} \hbar \omega_0\right)^2
  \\
  &= \left(\frac{5}{2} - \frac{9}{4}\right) (\hbar\omega_0)^2
  = \frac{1}{4} (\hbar\omega_0)^2
\end{align}

La déviation standard de $H$ est donc
\begin{align}
  \Delta H = \frac{1}{2} \hbar \omega_0
\end{align}


\subproblem
On constate que $\ket{\psi(0)}$ correspond à un état propre de $A$
  avec valeur propre $a$ car
\begin{align}
  A \ket{\psi(0)}
  =
  a
  \begin{pmatrix}
     1 & 0 & 0 \\
     0 & 0 & 1 \\
     0 & 1 & 0 \\
  \end{pmatrix}
  \cdot
  \begin{pmatrix}
     \tfrac{1}{\sqrt{2}} \\
     \tfrac{1}{2} \\
     \tfrac{1}{2} \\
  \end{pmatrix}
  =
  a
  \begin{pmatrix}
     \tfrac{1}{\sqrt{2}} \\
     \tfrac{1}{2} \\
     \tfrac{1}{2} \\
  \end{pmatrix}
  = a \ket{\psi(0)}
\end{align}
Une mesure de $A$ donne donc une valeur de $a$
  avec une probabilité de 1,
  et après la mesure, le système est encore dans l'état $\ket{\psi(0)}$.

\subproblem
Au temps $t$, l'état du système évoluera selon
\begin{align}
  \ket{\psi(t)}
  &= \frac{e^{-iE_1 t}}{\sqrt{2}} \ket{u_1}
   + \frac{e^{-iE_2 t}}{2} \ket{u_2}
   + \frac{e^{-iE_3 t}}{2} \ket{u_3}
\\
  &= \frac{e^{-i\hbar \omega_0 t}}{\sqrt{2}} \ket{u_1}
   + \frac{e^{-i2\hbar \omega_0 t}}{2} \ket{u_2}
   + \frac{e^{-i2\hbar \omega_0 t}}{2} \ket{u_3}
\end{align}


\subproblem
On remarque que les opérateurs $H$ et $A$ commutent.
Donc, $\ket{\psi(t)}$ est encore un état propre de $A$ avec valeur propre $a$,
  c'est à dire que
\begin{align}
  \braket{A}(t) = a
\end{align}

Par contre, $B$ ne commute pas avec $H$, et on peut calculer
\begin{align}
  \braket{B}(t)
  &=
  \braket{\psi(t) \vert B \vert \psi(t)}
  \\
  &=
  b
  \begin{pmatrix}
  e^{i\hbar \omega_0 t} / \sqrt{2}, &
  e^{i2\hbar \omega_0 t} / 2, &
  e^{i2\hbar \omega_0 t} / 2 
  \end{pmatrix}
  \cdot
  \begin{pmatrix}
     0 & 1 & 0 \\
     1 & 0 & 0 \\
     0 & 0 & 1 \\
  \end{pmatrix}
  \cdot
  \begin{pmatrix}
  e^{-i\hbar \omega_0 t} / \sqrt{2} \\
  e^{-i2\hbar \omega_0 t} / 2 \\
  e^{-i2\hbar \omega_0 t} / 2 \\
  \end{pmatrix}
  \\
  &=
  b
  \begin{pmatrix}
  e^{i\hbar \omega_0 t} / \sqrt{2}, &
  e^{i2\hbar \omega_0 t} / 2, &
  e^{i2\hbar \omega_0 t} / 2 
  \end{pmatrix}
  \cdot
  \begin{pmatrix}
  e^{-i2\hbar \omega_0 t} / 2 \\
  e^{-i\hbar \omega_0 t} / \sqrt{2} \\
  e^{-i2\hbar \omega_0 t} / 2 \\
  \end{pmatrix}
  \\
  &=
  b
  \left(
  \frac{e^{-i\hbar \omega_0 t}}{ 2 \sqrt{2}}
  +
  \frac{e^{ i\hbar \omega_0 t}}{ 2 \sqrt{2}}
  +
  \frac{1}{4}
  \right)
  \\
  &=
  b
  \left(
  \frac{\cos(\hbar \omega_0 t)}{\sqrt{2}}
  +
  \frac{1}{4}
  \right)
\end{align}

La valeur moyenne de $B$ oscille dans le temps
  car en général, l'état $\ket{\psi(t)}$ n'est pas un état propre de $B$.


\subproblem
Comme l'état $\ket{\psi(t)}$ est toujours un état propre de $A$,
  au temps $t$, une mesure de $A$ donne toujours la valeur $a$.
%

Une mesure de l'opérateur $B$ au temps $t$
  peut donner les différentes valeurs propres de $B$,
  soit $b$ ou $-b$,
  puisque la fonction d'onde $\ket{\psi(t)}$ peut avoir une projection
  non-nulle selon les états propres de $B$ correspondants.





\end{solution}
\end{problem}

\end{document}
